\documentclass[12pt]{amsart}

% this is your DRP LaTeX template! this first header section is just full of 
% technical prerequisites to make sure all the packages you need are included,
% all the various layout and formatting details are pretty, etc. etc.

\makeatletter
\def\@settitle{\begin{center}%
		\baselineskip14\p@\relax
		%\bfseries
		\normalfont\Large%<- NEW
		\@title
	\end{center}%
}
\makeatletter
\def\@settitle{\begin{center}%
		\baselineskip14\p@\relax
		%\bfseries
		\normalfont\Large%<- NEW
		\@title
	\end{center}%
}
\makeatother

\usepackage{graphicx,epsfig}
\usepackage{bbm}
\usepackage{esvect}
\usepackage{tikz-cd}
\usepackage{fancyhdr}
\usepackage{amsfonts}
\usepackage{amsmath}
\usepackage{amssymb}
\usepackage{mathabx}
\usepackage{amsthm}
\usepackage{tikz-cd}
\usepackage{parskip}
\usepackage{times} 
\usepackage{mathrsfs}
\usepackage{latexsym}
\usepackage{amscd}
\usepackage{xy}
\input{xy}
\xyoption{all}

\makeindex \setcounter{tocdepth}{2}


\voffset = -20pt
\hoffset = -80pt
\textwidth = 520pt \textheight
=650pt \headheight = 8pt \headsep = 10pt

%References as links
\usepackage[
bibstyle=alphabetic,
citestyle=alphabetic,
hyperref=true,
backref=false]{biblatex}
\usepackage{hyperref}
\hypersetup{pdftoolbar=true, pdftitle={GKform}, pdffitwindow=true, colorlinks=true, citecolor=green, filecolor=black, linkcolor=blue, urlcolor=blue, hypertexnames=false}
\usepackage{cleveref}
\makeindex \setcounter{tocdepth}{2}

\voffset = -20pt
\hoffset = -80pt
\textwidth = 520pt \textheight
=650pt \headheight = 8pt \headsep = 10pt

% this is the link to the file which has your bibliography information!
\addbibresource{drp.bib}

% defining some common macros (shortcuts for commands in this document)
% for example, below, the first line lets us write \bbR instead of \mathbb{R}
% for the symbol commonly used for the real numbers. based on these,
% you can define your own macros if you're feeling ambitious!
\newcommand{\bbR}{\ensuremath{\mathbb{R}}}
\newcommand{\bbZ}{\ensuremath{\mathbb{Z}}}
\newcommand{\bbQ}{\ensuremath{\mathbb{Q}}}

\newtheorem{thm}{Theorem}[section]
\newtheorem{prop}[thm]{Proposition}
\newtheorem{lem}[thm]{Lemma}
\newtheorem{cor}[thm]{Corollary}
\newtheorem{conj}[thm]{Conjecture}
\newtheorem{defn}[thm]{Definition}
\theoremstyle{definition}
\newtheorem{rem}[thm]{Remark}
\newtheorem{ex}[thm]{Example}
\numberwithin{equation}{section}

% this is where your information goes!
\title{Title goes here}
\author{DRP student}
\date{Date goes here}
\renewcommand{\baselinestretch}{1.3} 
\renewcommand{\headrulewidth}{0pt}

\fancyhead{}
\pagestyle{fancy}

\makeatletter
\let\Author\author
\let\Title\@title
\makeatother
\begin{document}

\maketitle

\begin{abstract}
	If you want an abstract, you can put it here! Otherwise, just delete this part.
\end{abstract}

\tableofcontents

\section{Introduction}

Here's your first section! You can write proofs and equations, like $x+y^2=3\sigma_3(t)$. You can even center them, for example
\[
\left( \sum_{i=1}^n a_i \right)\left(\prod_{j=1}^m b_j\right) = 1.
\]
Another useful trick is the align environment, which allows you to produce results like the following:
\begin{align}
3x+4 & = 3(y+3)+4 \\
& = 3y + 9 +4 \\
& = 3y + 13
\end{align}

\subsection{A subsection}

You can define \emph{subsections} for better organization. Here's a diagram using the package \texttt{tikzcd}:
\[
\begin{tikzcd}
A \arrow[r] \arrow[d] & B \arrow[d]\\
C \arrow[r] & D
\end{tikzcd}
\]
You can also add \textbf{citations}!\footnote{And footnotes are also possible, if you like those.} For example, \cite[\S 15]{Abbes} is a citation which works because there is a corresponding entry ``Abbes'' in the companion \texttt{drp.bib} file which was included in the header. In that file, you will find several different examples of references you can add, which will be printed at the end of the document.

\subsubsection{A subsubsection.}

Though probably not necessary, it's also even possible to define subsubsections.

\section{Another section}

Here's another section!

% this prints the list of references from drp.bib which were used in this document
\printbibliography

\end{document}